% GA4GH File Encryption Standard specification
% When editing, please start all sentences on a new line; and try to avoid source lines longer than 120 characters.
\documentclass[10pt]{article}
\usepackage[margin=1in]{geometry}
\usepackage[pdfborder={0 0 0},hyperfootnotes=false]{hyperref}
\usepackage{parskip}
\newcommand{\mailtourl}[1]{\href{mailto:#1}{\tt #1}}
\newcommand{\tagvalue}[1]{\tt #1}
\newcommand{\tagregex}[1]{\tt #1}
\begin{document}

% GA4GH File Encryption Standard specification
% When editing, please start all sentences on a new line; and try to avoid source lines longer than 120 characters.
\documentclass[10pt]{article}
\usepackage[margin=1in]{geometry}
\usepackage[pdfborder={0 0 0},hyperfootnotes=false]{hyperref}
\usepackage{parskip}
\usepackage{draftwatermark}
\newcommand{\mailtourl}[1]{\href{mailto:#1}{\tt #1}}
\newcommand{\tagvalue}[1]{\tt #1}
\newcommand{\tagregex}[1]{\tt #1}
\newcommand{\kw}[1]{\texttt{#1}}

\SetWatermarkAngle{315}
\SetWatermarkText{PRELIMINARY}
\SetWatermarkScale{4}

\begin{document}

% GA4GH File Encryption Standard specification
% When editing, please start all sentences on a new line; and try to avoid source lines longer than 120 characters.
\documentclass[10pt]{article}
\usepackage[margin=1in]{geometry}
\usepackage[pdfborder={0 0 0},hyperfootnotes=false]{hyperref}
\usepackage{parskip}
\usepackage{draftwatermark}
\newcommand{\mailtourl}[1]{\href{mailto:#1}{\tt #1}}
\newcommand{\tagvalue}[1]{\tt #1}
\newcommand{\tagregex}[1]{\tt #1}
\newcommand{\kw}[1]{\texttt{#1}}

\SetWatermarkAngle{315}
\SetWatermarkText{PRELIMINARY}
\SetWatermarkScale{4}

\begin{document}

% GA4GH File Encryption Standard specification
% When editing, please start all sentences on a new line; and try to avoid source lines longer than 120 characters.
\documentclass[10pt]{article}
\usepackage[margin=1in]{geometry}
\usepackage[pdfborder={0 0 0},hyperfootnotes=false]{hyperref}
\usepackage{parskip}
\usepackage{draftwatermark}
\newcommand{\mailtourl}[1]{\href{mailto:#1}{\tt #1}}
\newcommand{\tagvalue}[1]{\tt #1}
\newcommand{\tagregex}[1]{\tt #1}
\newcommand{\kw}[1]{\texttt{#1}}

\SetWatermarkAngle{315}
\SetWatermarkText{PRELIMINARY}
\SetWatermarkScale{4}

\begin{document}

\input{crypt4gh.ver}
\title{GA4GH File Encryption Standard}
\date{\headdate}
\maketitle
\begin{quote}\small
The master version of this document can be found at
\url{https://github.com/samtools/hts-specs}.\\
This printing is version~\commitdesc\ from that repository,
last modified on the date shown above.
\end{quote}
\vspace*{1em}

\begin{abstract}
This document describes the format for Global Alliance for Genomics and Health (GA4GH) encrypted and authenticated
files.
Encryption helps to prevent accidental disclosure of confidential information.
Allowing programs to directly read and write data in an encrypted format reduces the chance of such disclosure.
The format described here can be used to encrypt any underlying file format.
It also allows for seeking on the encrypted data.
In particular indexes on the plain text version can also be used on the encrypted file without modification.
\end{abstract}
\newpage
\tableofcontents
\newpage
\section{Introduction}
\subsection{Purpose}
By its nature, genomic data can include information of a confidential nature about the health of individuals.
It is important that such information is not accidentally disclosed.
One part of the defence against such disclosure is to, as much as possible, keep the data in an encrypted format.

This document describes a file format that can be used to store data in an encrypted and authenticated state.
Existing applications can, with minimal modification, read and write data in the encrypted format.
The choice of encryption also allows the encrypted data to be read starting from any location, facilitating indexed
access to files.

\subsection{Requirements}
The key words ``MUST'', ``MUST NOT'', ``REQUIRED'', ``SHALL'', ``SHALL NOT'', ``SHOULD'', ``SHOULD NOT'',
``RECOMMENDED'', ``MAY'', and ``OPTIONAL'' in this document are to be interpreted as described in \cite{RFC2119}.

\subsection{Terminology}
\begin{description}
\item[Elliptic-curve cryptography (ECC)]
An approach to public-key cryptography based on the algebraic structure of elliptic curves over finite fields.
\item[Curve25519]
A widely used FIPS-140 approved ECC algorithm not encumbered by any patents \cite{RFC7748}.
\item[Ed25519]
An Edwards-curve Digital Signature Algorithm (EdDSA) over Curve25519 \cite{RFC8032}.
\item[ChaCha20-IETF-Poly1305]
ChaCha20 is a symmetric stream cipher built on a pseudo-random function that gives the advantage that one can
efficiently seek to any position in the key stream in constant time.
It is not patented.
Poly1305 is a cryptographic message authentication code (MAC).
It can be used to verify the data integrity and the authenticity of a message \cite{RFC8439}.

\item[cipher-text]~\\
The encrypted version of the data.

\item[plain-text]~\\
The unencrypted version of the data.

\end{description}

\section{Encrypted Representation Overview}
The encrypted file consists of three parts:

\begin{itemize}
\item An unencrypted header, containing a magic number, version number, and parameters needed to decode the encrypted
header.

\item An encrypted header, which is encrypted using an asymmetric encryption algorithm.
It lists the encryption key and other parameters needed to decrypt the encrypted data section.

\item The encrypted data.
This is the actual application data.
It is encrypted using a symmetric encryption algorithm as described in the encrypted header.
\end{itemize}

\section{Detailed Specification}
\subsection{Overall Conventions}
\subsubsection{Hexadecimal Numbers}
Hexadecimal values are written using the digits 0-9, and letters a-f for values 10-15.
Values are written with the most-significant digit on the left, and prefixed with "0x".

\subsubsection{Byte Ordering}
The basic data size is the byte (8 bits).
All multi-byte vales are stored in least-significant byte first (``little-endian'') order.
For example, the value 1234 decimal (0x4d2) is stored as the byte stream 0xd2 0x04.

\subsubsection{Integer Types}
Integers can be either signed or unsigned.
Signed values are stored in two's complement form.

\subsubsection{Multi-byte Integer Types}
\begin{center}
\begin{tabular}{l l l l}
\hline
\textbf{Name} & \textbf{Byte Ordering} & \textbf{Integer Type} & \textbf{Size (bytes)} \\
\hline
byte & & unsigned & 1 \\
le\_int32 & little-endian & signed & 4 \\
le\_uint32 & little-endian & unsigned & 4 \\
le\_int64 & little-endian & signed & 8 \\
le\_uint64 & little-endian & unsigned & 8 \\
le\_uint96 & little-endian & unsigned & 12 \\
\end{tabular}
\end{center}

\subsubsection{Structures}
Structure types may be defined (in C-like notation) for convenience.

\begin{verbatim}
struct demo {
  byte string[8];
  le_int32 number1;
  le_uint64 number2;
};
\end{verbatim}

When structures are serialized to a file, elements are written in the given order with no padding between them.
The above structure would be written as twenty bytes - eight for the array `string', four for the integer `number1',
and eight for the integer `number2'.

\subsubsection{Enumerated Types}
Enumerated types may only take one of a given set of values.
The data type used to store the enumerated value is given in angle brackets after the type name.
Every element of an enumerated type must be assigned a value.
It is not valid to compare values between two enumerated types.

\begin{verbatim}
enum Animal<le_uint32> {
  cat    = 1;
  dog    = 2;
  rabbit = 3;
};
\end{verbatim}

\subsubsection{Variants}
Parts of structures may vary depending on information available at the time of decoding.
Which variant to use is selected by an enumerated type.
There must be a case for every possible enumerated value.
Cases have limited fall-through.
Consecutive cases with no fields in between all contain the same fields.

\begin{verbatim}
struct AnimalFeatures {
  select (enum Animal) {
    case cat:
    case dog:
      le_uint32 hairyness;
      le_uint32 whisker_length;

    case rabbit:
      le_uint32 ear_length;
  };
};
\end{verbatim}

For the `cat' and `dog' cases, `struct AnimalFeatures' is eight bytes long and contains two unsigned four-byte
little-endian values.
For the `rabbit' case it is four bytes long and contains a single four-byte little-endian value.

If the cases are different lengths (as above), then the size of the overall structure depends on the variant chosen.
There is NO padding to make the cases the same length unless it is explicitly defined.

\subsection{Header}

The file starts with an header, with the following structure:

\begin{verbatim}
struct Header {
  byte                          magic_number[8];
  le_uint32                     version;
  le_uint32                     header_length;
  Header_encryption_parameters  header_encryption;
  byte[]                        encrypted_header
};
\end{verbatim}

The \kw{magic\_number} is the ASCII representation of the string ``crypt4gh''.

The version number is stored as a four-byte little-endian unsigned integer.
The current version number is 1.

The current byte representation of the magic number and version is:
\begin{verbatim}
0x63 0x72 0x79 0x70 0x74 0x34 0x67 0x68 0x01 0x00 0x00 0x00
============= magic_number============= ===== version =====
\end{verbatim}

\kw{header\_length} is the combined length of the \kw{header\_encryption} and \kw{encrypted\_header} sections.

\subsubsection{Header Encryption Parameters}\label{header:encryption}


The `Header\_encryption\_parameters' type is defined as:

\begin{verbatim}

struct Header_encryption_parameters {
  enum Header_encryption_method<le_uint32> header_encryption_method;

  select (header_encryption_method) {
    case X25519_chacha20_ietf_poly1305:
      byte                         public_key[32];
      byte                         nonce[12];
  };
};
\end{verbatim}

\kw{header\_encryption\_method} is an enumerated type that describes the type of encryption to be used.

\kw{public\_key} is the writer's public key.
Treated as a concatenation of eight 32-bit little-endian integers.

\kw{nonce} is a unique initialization vector.
In chacha20\_ietf\_poly1305 it is 12 bytes long.
This value must be unique for each file shared between the same reader and writer.
The best way to ensure this is to generate a value with a cryptographically-secure random number generator.

\subsubsection{Encrypted Header Format}

The encrypted header encodes the parameters used to read the data part of the file.
The plain-text is defined as:

\begin{verbatim}
enum Data_encryption_method<le_uint32> {
  chacha20_ietf_poly1305 = 0;
};

struct EncryptionParameters {
  enum EncryptionMethod<le_uint32>  data_encryption_method;
  select (method) {
    case chacha20_ietf_poly1305:
      byte                          key[32];
  };
};
\end{verbatim}

\kw{data\_encryption\_method} is an enumerated type that describes the type of encryption to be used.

\kw{key} is the symmetric key used to decode the data section.

\subsubsection{X25519\_chacha20\_ietf\_poly1305 Encryption}

This method uses Diffie-Hellman key exchange with additional hashing to generate a shared key.
The shared key is then used along with a randomly-generated nonce to encrypt the data using the chacha20\_ietf\_poly1305
construction.

Encryption requires the writer's public and secret keys ($K_{pw}$ and $K_{sw}$), the reader's public key ($K_{pr}$)
and the nonce ($N$).

The writer's secret key and the reader's public key are used to generate a Diffie\_Hellman shared key using the
X25519 algorithm described in section 5 of \cite{RFC7748}.

$K_{dh} = X25519(K_{sw}, K_{pr})$

As the X25519 algorithm does not produce a completely uniform bit distribution, and many possible $(K_{sw}, K_{pr})$
pairs can produce the same output, the Diffie-Hellman key is hashed along with the two public keys to produce the
final shared key.
The hash function used to do this is Blake2b, as described in \cite{RFC7693}.

$K_s = Blake2b(K_{dh} || K_{pr} || K_{pw})$

As ChaCha20 uses a 32-byte key, only the first 32 bytes of $K_s$ are used; the rest are discarded.

The header data is then encrypted using the method described in the ChaCha20\_IETF\_Poly1305 Encryption
section~\ref{data:chacha20_encryption}.

Finally, the writer's public key $K_{pw}$, the nonce $N$ and the encrypted header data are combined into
the file header.

\subsubsection{X25519\_chacha20\_ietf\_poly1305 Decryption}

To decrypt the header, the reader needs the writer's public key $K_{pw}$ and the nonce from the file header.
Also needed are the reader's public and secret keys ($K_{pr}$ and $K_{sr}$).

The Diffie-Helman key is obtained using:

$K_{dh} = X25519(K_{sr}, K_{pw})$

This is then hashed to obtain the shared key (again only the first 32 bytes are retained):

$K_s = Blake2b(K_{dh} || K_{pr} || K_{pw})$

The resulting key $K_s$ and nonce $N$ are then used to decrypt the remainder of the header.

\subsection{Encrypted Data}
\subsubsection{ChaCha20\_IETF\_Poly1305 Encryption}\label{data:chacha20_encryption}

ChaCha20 is a stream cipher which maps a 256-bit key, nonce and counter to a 512-bit keystream block.
In IETF mode the nonce is 96 bits long and the counter is 32 bits.
The counter starts at 1, and is incremented by 1 for each successive keystream block.
The cipher-text is the plain-text message combined with the keystream using the bitwise exclusive-or operation.

Poly1305 is used to generate a 16-byte message authentication code (MAC) over the cipher-text.
As the MAC is generated over the entire cipher-text it is not possible to authenticate partially decrypted data.

ChaCha20 and Poly1305 are combined using the AEAD construction described in section 2.8 of \cite{RFC8439}.
This construction allows additional authenticated data (AAD) to be included in the Poly1305 MAC calculation.
For the purposes of this format, the AAD is zero bytes long.

\subsubsection{Segmenting the input}

To allow random access without having to authenticate the entire file, the plain-text is divided into 65536-byte
(64KiB) segments.
If the plain-text is not a multiple of 64KiB long, the last segment will be shorter.
Each segment is encrypted using the method defined in the header.
The nonce used to encrypt the segment is then stored, followed by the encrypted data, and then the MAC.

\begin{verbatim}
struct Segment {
  select (method) {
    case chacha20_ietf_poly1305:
      byte    nonce[12];
      byte[]  encrypted_data;
      byte    mac[16];
  };
};
\end{verbatim}

The addition of the \kw{nonce} and \kw{mac} bytes will expand the data slightly.
For chacha20\_ietf\_poly1305, this expansion will be 28 bytes, so a 65536 byte plain-text input will become a 65564
byte encrypted and authenticated cipher-text output.

\section{Decryption}

The cipher-text is decrypted by authenticating and decrypting the segment(s) enclosing the requested byte range $[P;Q]$,
where $P<Q$.
For a range starting at position $P$, the location of the segment \kw{seg\_start} containing that position must
first be found.
For the chacha20\_ietf\_poly1305 method, this can be done using the formula:

\begin{verbatim}
seg_start = header_len + floor(P/65536) * 65564
\end{verbatim}

For an encrypted segment starting at position \kw{seg\_start}, the nonce, then the 65536 bytes of ciphertext
(possibly fewer of it was the last segment), and finally the MAC are read.

An authentication tag is calculated over the ciphertext from that segment, and bitwise compared to the MAC.
The ciphertext is authenticated if and only if the tags match.
An error MUST be reported if the ciphertext is not authenticated.

The key and nonce are then used to decrypt the cipher-text for the segment, returning the plain-text.
Successive segments are decrypted, until the segment containing position $Q$ is reached.
The plaintext segments are concatenated to form the resulting output, discarding $P \mathbin{\%} 65536$ bytes from
the beginning of the first segment and retaining $Q \mathbin{\%} 65536$ bytes of the last one.

\section{Security Considerations}
\subsection{Threat Model}
This format is designed to protect files at rest and in transport from accidental disclosure.
Using authenticated encryption in individual segments mirrors solutions like Transport Layer
Security (TLS) as described in~\cite{RFC5246} and prevents undetected modification of segments.

\subsection{Selection of Key}
The security of the format depends on attackers not being able to guess the encryption key.
The encryption key MUST be generated using a cryptographically-secure pseudo-random number generator.
This makes the chance of guessing a key vanishingly small.

\subsection{Nonce selection}
All segments encrypted with the same key MUST use a unique nonce.
One way to ensure this is to choose a random starting point and then use a counter or linear-feedback shift
register to generate the nonce for each segment.
This method guarantees a unique value even for very long files.

For a 96-bit nonce, it may be acceptable to generate each one using a cryptographically-secure pseudo-random number
generator.
Care should be taken to ensure that the random number generator is capable of generating a long enough stream
of unique values.
Due to the birthday problem, this method will have a non-zero (but very small) probability of failing.
For example, a file of 24 Terabytes will have a reused nonce with probability of approximately $10^-12$

\subsection{Message Forgery}
Using Curve25519 and Ed25519 authenticates the content of the encrypted file header.
Using ChaCha20-ietf-Poly1305 authenticates the content of each segment of the encrypted cipher-text.

\subsection{No File Updates Permitted}
Implementations MUST NOT update encrypted files.
Once written, a section of the file must never be altered.

\addcontentsline{toc}{section}{References}
\bibliographystyle{abstract}

\renewcommand{\refname}{\section{References}} % Get References numbered correctly in the table of contents
\begin{thebibliography}{XXXXXXX}

% Normative references

\bibitem[RFC2119]{RFC2119}
  Bradner, S.,\\
  \emph{"Key words for use in RFCs to Indicate Requirement Levels", BCP 14, RFC 2119},\\
  \url{https://www.rfc-editor.org/info/rfc2119},\\
  March 1997

\bibitem[RFC7693]{RFC7693}
  Saarinen, M.J., Aumasson, J.P.\\
  \emph{"The BLAKE2 Cryptographic Hash and Message Authentication Code (MAC)", RFC7693},\\
  \url{https://tools.ietf.org/html/rfc7693},\\
  November 2015

\bibitem[RFC7748]{RFC7748}
  A. Langley, Google, M. Hamburg, Rambus Cryptography Research, S. Turner, sn3rd,\\
  \emph{"Elliptic Curves for Security", RFC7748},\\
  \url{https://tools.ietf.org/html/rfc7748},\\
  January 2016

\bibitem[RFC8032]{RFC8032}
  S. Josefsson, SJD AB, I. Liusvaara,\\
  \emph{"Edwards-Curve Digital Signature Algorithm (EdDSA)", RFC8032},\\
  \url{https://tools.ietf.org/html/rfc8032},\\
  January 2017

\bibitem[RFC8439]{RFC8439}
  Y. Nir, Dell EMC, A. Langley, Google, Inc.,\\
  \emph{"ChaCha20 and Poly1305 for IETF Protocols", RFC8439},\\
  \url{https://tools.ietf.org/html/rfc8439},\\
  June 2018

% Informational references

\bibitem[RFC5246]{RFC5246}
  Dierks, T., Rescorla, E.,\\
  \emph{"The Transport Layer Security (TLS) Protocol Version 1.2", RFC 5246},\\
  \url{https://www.rfc-editor.org/info/rfc5246},\\
  August 2008.

\end{thebibliography}

\end{document}

\title{GA4GH File Encryption Standard}
\date{\headdate}
\maketitle
\begin{quote}\small
The master version of this document can be found at
\url{https://github.com/samtools/hts-specs}.\\
This printing is version~\commitdesc\ from that repository,
last modified on the date shown above.
\end{quote}
\vspace*{1em}

\begin{abstract}
This document describes the format for Global Alliance for Genomics and Health (GA4GH) encrypted and authenticated
files.
Encryption helps to prevent accidental disclosure of confidential information.
Allowing programs to directly read and write data in an encrypted format reduces the chance of such disclosure.
The format described here can be used to encrypt any underlying file format.
It also allows for seeking on the encrypted data.
In particular indexes on the plain text version can also be used on the encrypted file without modification.
\end{abstract}
\newpage
\tableofcontents
\newpage
\section{Introduction}
\subsection{Purpose}
By its nature, genomic data can include information of a confidential nature about the health of individuals.
It is important that such information is not accidentally disclosed.
One part of the defence against such disclosure is to, as much as possible, keep the data in an encrypted format.

This document describes a file format that can be used to store data in an encrypted and authenticated state.
Existing applications can, with minimal modification, read and write data in the encrypted format.
The choice of encryption also allows the encrypted data to be read starting from any location, facilitating indexed
access to files.

\subsection{Requirements}
The key words ``MUST'', ``MUST NOT'', ``REQUIRED'', ``SHALL'', ``SHALL NOT'', ``SHOULD'', ``SHOULD NOT'',
``RECOMMENDED'', ``MAY'', and ``OPTIONAL'' in this document are to be interpreted as described in \cite{RFC2119}.

\subsection{Terminology}
\begin{description}
\item[Elliptic-curve cryptography (ECC)]
An approach to public-key cryptography based on the algebraic structure of elliptic curves over finite fields.
\item[Curve25519]
A widely used FIPS-140 approved ECC algorithm not encumbered by any patents \cite{RFC7748}.
\item[Ed25519]
An Edwards-curve Digital Signature Algorithm (EdDSA) over Curve25519 \cite{RFC8032}.
\item[ChaCha20-IETF-Poly1305]
ChaCha20 is a symmetric stream cipher built on a pseudo-random function that gives the advantage that one can
efficiently seek to any position in the key stream in constant time.
It is not patented.
Poly1305 is a cryptographic message authentication code (MAC).
It can be used to verify the data integrity and the authenticity of a message \cite{RFC8439}.

\item[cipher-text]~\\
The encrypted version of the data.

\item[plain-text]~\\
The unencrypted version of the data.

\end{description}

\section{Encrypted Representation Overview}
The encrypted file consists of three parts:

\begin{itemize}
\item An unencrypted header, containing a magic number, version number, and parameters needed to decode the encrypted
header.

\item An encrypted header, which is encrypted using an asymmetric encryption algorithm.
It lists the encryption key and other parameters needed to decrypt the encrypted data section.

\item The encrypted data.
This is the actual application data.
It is encrypted using a symmetric encryption algorithm as described in the encrypted header.
\end{itemize}

\section{Detailed Specification}
\subsection{Overall Conventions}
\subsubsection{Hexadecimal Numbers}
Hexadecimal values are written using the digits 0-9, and letters a-f for values 10-15.
Values are written with the most-significant digit on the left, and prefixed with "0x".

\subsubsection{Byte Ordering}
The basic data size is the byte (8 bits).
All multi-byte vales are stored in least-significant byte first (``little-endian'') order.
For example, the value 1234 decimal (0x4d2) is stored as the byte stream 0xd2 0x04.

\subsubsection{Integer Types}
Integers can be either signed or unsigned.
Signed values are stored in two's complement form.

\subsubsection{Multi-byte Integer Types}
\begin{center}
\begin{tabular}{l l l l}
\hline
\textbf{Name} & \textbf{Byte Ordering} & \textbf{Integer Type} & \textbf{Size (bytes)} \\
\hline
byte & & unsigned & 1 \\
le\_int32 & little-endian & signed & 4 \\
le\_uint32 & little-endian & unsigned & 4 \\
le\_int64 & little-endian & signed & 8 \\
le\_uint64 & little-endian & unsigned & 8 \\
le\_uint96 & little-endian & unsigned & 12 \\
\end{tabular}
\end{center}

\subsubsection{Structures}
Structure types may be defined (in C-like notation) for convenience.

\begin{verbatim}
struct demo {
  byte string[8];
  le_int32 number1;
  le_uint64 number2;
};
\end{verbatim}

When structures are serialized to a file, elements are written in the given order with no padding between them.
The above structure would be written as twenty bytes - eight for the array `string', four for the integer `number1',
and eight for the integer `number2'.

\subsubsection{Enumerated Types}
Enumerated types may only take one of a given set of values.
The data type used to store the enumerated value is given in angle brackets after the type name.
Every element of an enumerated type must be assigned a value.
It is not valid to compare values between two enumerated types.

\begin{verbatim}
enum Animal<le_uint32> {
  cat    = 1;
  dog    = 2;
  rabbit = 3;
};
\end{verbatim}

\subsubsection{Variants}
Parts of structures may vary depending on information available at the time of decoding.
Which variant to use is selected by an enumerated type.
There must be a case for every possible enumerated value.
Cases have limited fall-through.
Consecutive cases with no fields in between all contain the same fields.

\begin{verbatim}
struct AnimalFeatures {
  select (enum Animal) {
    case cat:
    case dog:
      le_uint32 hairyness;
      le_uint32 whisker_length;

    case rabbit:
      le_uint32 ear_length;
  };
};
\end{verbatim}

For the `cat' and `dog' cases, `struct AnimalFeatures' is eight bytes long and contains two unsigned four-byte
little-endian values.
For the `rabbit' case it is four bytes long and contains a single four-byte little-endian value.

If the cases are different lengths (as above), then the size of the overall structure depends on the variant chosen.
There is NO padding to make the cases the same length unless it is explicitly defined.

\subsection{Header}

The file starts with an header, with the following structure:

\begin{verbatim}
struct Header {
  byte                          magic_number[8];
  le_uint32                     version;
  le_uint32                     header_length;
  Header_encryption_parameters  header_encryption;
  byte[]                        encrypted_header
};
\end{verbatim}

The \kw{magic\_number} is the ASCII representation of the string ``crypt4gh''.

The version number is stored as a four-byte little-endian unsigned integer.
The current version number is 1.

The current byte representation of the magic number and version is:
\begin{verbatim}
0x63 0x72 0x79 0x70 0x74 0x34 0x67 0x68 0x01 0x00 0x00 0x00
============= magic_number============= ===== version =====
\end{verbatim}

\kw{header\_length} is the combined length of the \kw{header\_encryption} and \kw{encrypted\_header} sections.

\subsubsection{Header Encryption Parameters}\label{header:encryption}


The `Header\_encryption\_parameters' type is defined as:

\begin{verbatim}

struct Header_encryption_parameters {
  enum Header_encryption_method<le_uint32> header_encryption_method;

  select (header_encryption_method) {
    case X25519_chacha20_ietf_poly1305:
      byte                         public_key[32];
      byte                         nonce[12];
  };
};
\end{verbatim}

\kw{header\_encryption\_method} is an enumerated type that describes the type of encryption to be used.

\kw{public\_key} is the writer's public key.
Treated as a concatenation of eight 32-bit little-endian integers.

\kw{nonce} is a unique initialization vector.
In chacha20\_ietf\_poly1305 it is 12 bytes long.
This value must be unique for each file shared between the same reader and writer.
The best way to ensure this is to generate a value with a cryptographically-secure random number generator.

\subsubsection{Encrypted Header Format}

The encrypted header encodes the parameters used to read the data part of the file.
The plain-text is defined as:

\begin{verbatim}
enum Data_encryption_method<le_uint32> {
  chacha20_ietf_poly1305 = 0;
};

struct EncryptionParameters {
  enum EncryptionMethod<le_uint32>  data_encryption_method;
  select (method) {
    case chacha20_ietf_poly1305:
      byte                          key[32];
  };
};
\end{verbatim}

\kw{data\_encryption\_method} is an enumerated type that describes the type of encryption to be used.

\kw{key} is the symmetric key used to decode the data section.

\subsubsection{X25519\_chacha20\_ietf\_poly1305 Encryption}

This method uses Diffie-Hellman key exchange with additional hashing to generate a shared key.
The shared key is then used along with a randomly-generated nonce to encrypt the data using the chacha20\_ietf\_poly1305
construction.

Encryption requires the writer's public and secret keys ($K_{pw}$ and $K_{sw}$), the reader's public key ($K_{pr}$)
and the nonce ($N$).

The writer's secret key and the reader's public key are used to generate a Diffie\_Hellman shared key using the
X25519 algorithm described in section 5 of \cite{RFC7748}.

$K_{dh} = X25519(K_{sw}, K_{pr})$

As the X25519 algorithm does not produce a completely uniform bit distribution, and many possible $(K_{sw}, K_{pr})$
pairs can produce the same output, the Diffie-Hellman key is hashed along with the two public keys to produce the
final shared key.
The hash function used to do this is Blake2b, as described in \cite{RFC7693}.

$K_s = Blake2b(K_{dh} || K_{pr} || K_{pw})$

As ChaCha20 uses a 32-byte key, only the first 32 bytes of $K_s$ are used; the rest are discarded.

The header data is then encrypted using the method described in the ChaCha20\_IETF\_Poly1305 Encryption
section~\ref{data:chacha20_encryption}.

Finally, the writer's public key $K_{pw}$, the nonce $N$ and the encrypted header data are combined into
the file header.

\subsubsection{X25519\_chacha20\_ietf\_poly1305 Decryption}

To decrypt the header, the reader needs the writer's public key $K_{pw}$ and the nonce from the file header.
Also needed are the reader's public and secret keys ($K_{pr}$ and $K_{sr}$).

The Diffie-Helman key is obtained using:

$K_{dh} = X25519(K_{sr}, K_{pw})$

This is then hashed to obtain the shared key (again only the first 32 bytes are retained):

$K_s = Blake2b(K_{dh} || K_{pr} || K_{pw})$

The resulting key $K_s$ and nonce $N$ are then used to decrypt the remainder of the header.

\subsection{Encrypted Data}
\subsubsection{ChaCha20\_IETF\_Poly1305 Encryption}\label{data:chacha20_encryption}

ChaCha20 is a stream cipher which maps a 256-bit key, nonce and counter to a 512-bit keystream block.
In IETF mode the nonce is 96 bits long and the counter is 32 bits.
The counter starts at 1, and is incremented by 1 for each successive keystream block.
The cipher-text is the plain-text message combined with the keystream using the bitwise exclusive-or operation.

Poly1305 is used to generate a 16-byte message authentication code (MAC) over the cipher-text.
As the MAC is generated over the entire cipher-text it is not possible to authenticate partially decrypted data.

ChaCha20 and Poly1305 are combined using the AEAD construction described in section 2.8 of \cite{RFC8439}.
This construction allows additional authenticated data (AAD) to be included in the Poly1305 MAC calculation.
For the purposes of this format, the AAD is zero bytes long.

\subsubsection{Segmenting the input}

To allow random access without having to authenticate the entire file, the plain-text is divided into 65536-byte
(64KiB) segments.
If the plain-text is not a multiple of 64KiB long, the last segment will be shorter.
Each segment is encrypted using the method defined in the header.
The nonce used to encrypt the segment is then stored, followed by the encrypted data, and then the MAC.

\begin{verbatim}
struct Segment {
  select (method) {
    case chacha20_ietf_poly1305:
      byte    nonce[12];
      byte[]  encrypted_data;
      byte    mac[16];
  };
};
\end{verbatim}

The addition of the \kw{nonce} and \kw{mac} bytes will expand the data slightly.
For chacha20\_ietf\_poly1305, this expansion will be 28 bytes, so a 65536 byte plain-text input will become a 65564
byte encrypted and authenticated cipher-text output.

\section{Decryption}

The cipher-text is decrypted by authenticating and decrypting the segment(s) enclosing the requested byte range $[P;Q]$,
where $P<Q$.
For a range starting at position $P$, the location of the segment \kw{seg\_start} containing that position must
first be found.
For the chacha20\_ietf\_poly1305 method, this can be done using the formula:

\begin{verbatim}
seg_start = header_len + floor(P/65536) * 65564
\end{verbatim}

For an encrypted segment starting at position \kw{seg\_start}, the nonce, then the 65536 bytes of ciphertext
(possibly fewer of it was the last segment), and finally the MAC are read.

An authentication tag is calculated over the ciphertext from that segment, and bitwise compared to the MAC.
The ciphertext is authenticated if and only if the tags match.
An error MUST be reported if the ciphertext is not authenticated.

The key and nonce are then used to decrypt the cipher-text for the segment, returning the plain-text.
Successive segments are decrypted, until the segment containing position $Q$ is reached.
The plaintext segments are concatenated to form the resulting output, discarding $P \mathbin{\%} 65536$ bytes from
the beginning of the first segment and retaining $Q \mathbin{\%} 65536$ bytes of the last one.

\section{Security Considerations}
\subsection{Threat Model}
This format is designed to protect files at rest and in transport from accidental disclosure.
Using authenticated encryption in individual segments mirrors solutions like Transport Layer
Security (TLS) as described in~\cite{RFC5246} and prevents undetected modification of segments.

\subsection{Selection of Key}
The security of the format depends on attackers not being able to guess the encryption key.
The encryption key MUST be generated using a cryptographically-secure pseudo-random number generator.
This makes the chance of guessing a key vanishingly small.

\subsection{Nonce selection}
All segments encrypted with the same key MUST use a unique nonce.
One way to ensure this is to choose a random starting point and then use a counter or linear-feedback shift
register to generate the nonce for each segment.
This method guarantees a unique value even for very long files.

For a 96-bit nonce, it may be acceptable to generate each one using a cryptographically-secure pseudo-random number
generator.
Care should be taken to ensure that the random number generator is capable of generating a long enough stream
of unique values.
Due to the birthday problem, this method will have a non-zero (but very small) probability of failing.
For example, a file of 24 Terabytes will have a reused nonce with probability of approximately $10^-12$

\subsection{Message Forgery}
Using Curve25519 and Ed25519 authenticates the content of the encrypted file header.
Using ChaCha20-ietf-Poly1305 authenticates the content of each segment of the encrypted cipher-text.

\subsection{No File Updates Permitted}
Implementations MUST NOT update encrypted files.
Once written, a section of the file must never be altered.

\addcontentsline{toc}{section}{References}
\bibliographystyle{abstract}

\renewcommand{\refname}{\section{References}} % Get References numbered correctly in the table of contents
\begin{thebibliography}{XXXXXXX}

% Normative references

\bibitem[RFC2119]{RFC2119}
  Bradner, S.,\\
  \emph{"Key words for use in RFCs to Indicate Requirement Levels", BCP 14, RFC 2119},\\
  \url{https://www.rfc-editor.org/info/rfc2119},\\
  March 1997

\bibitem[RFC7693]{RFC7693}
  Saarinen, M.J., Aumasson, J.P.\\
  \emph{"The BLAKE2 Cryptographic Hash and Message Authentication Code (MAC)", RFC7693},\\
  \url{https://tools.ietf.org/html/rfc7693},\\
  November 2015

\bibitem[RFC7748]{RFC7748}
  A. Langley, Google, M. Hamburg, Rambus Cryptography Research, S. Turner, sn3rd,\\
  \emph{"Elliptic Curves for Security", RFC7748},\\
  \url{https://tools.ietf.org/html/rfc7748},\\
  January 2016

\bibitem[RFC8032]{RFC8032}
  S. Josefsson, SJD AB, I. Liusvaara,\\
  \emph{"Edwards-Curve Digital Signature Algorithm (EdDSA)", RFC8032},\\
  \url{https://tools.ietf.org/html/rfc8032},\\
  January 2017

\bibitem[RFC8439]{RFC8439}
  Y. Nir, Dell EMC, A. Langley, Google, Inc.,\\
  \emph{"ChaCha20 and Poly1305 for IETF Protocols", RFC8439},\\
  \url{https://tools.ietf.org/html/rfc8439},\\
  June 2018

% Informational references

\bibitem[RFC5246]{RFC5246}
  Dierks, T., Rescorla, E.,\\
  \emph{"The Transport Layer Security (TLS) Protocol Version 1.2", RFC 5246},\\
  \url{https://www.rfc-editor.org/info/rfc5246},\\
  August 2008.

\end{thebibliography}

\end{document}

\title{GA4GH File Encryption Standard}
\date{\headdate}
\maketitle
\begin{quote}\small
The master version of this document can be found at
\url{https://github.com/samtools/hts-specs}.\\
This printing is version~\commitdesc\ from that repository,
last modified on the date shown above.
\end{quote}
\vspace*{1em}

\begin{abstract}
This document describes the format for Global Alliance for Genomics and Health (GA4GH) encrypted and authenticated
files.
Encryption helps to prevent accidental disclosure of confidential information.
Allowing programs to directly read and write data in an encrypted format reduces the chance of such disclosure.
The format described here can be used to encrypt any underlying file format.
It also allows for seeking on the encrypted data.
In particular indexes on the plain text version can also be used on the encrypted file without modification.
\end{abstract}
\newpage
\tableofcontents
\newpage
\section{Introduction}
\subsection{Purpose}
By its nature, genomic data can include information of a confidential nature about the health of individuals.
It is important that such information is not accidentally disclosed.
One part of the defence against such disclosure is to, as much as possible, keep the data in an encrypted format.

This document describes a file format that can be used to store data in an encrypted and authenticated state.
Existing applications can, with minimal modification, read and write data in the encrypted format.
The choice of encryption also allows the encrypted data to be read starting from any location, facilitating indexed
access to files.

\subsection{Requirements}
The key words ``MUST'', ``MUST NOT'', ``REQUIRED'', ``SHALL'', ``SHALL NOT'', ``SHOULD'', ``SHOULD NOT'',
``RECOMMENDED'', ``MAY'', and ``OPTIONAL'' in this document are to be interpreted as described in \cite{RFC2119}.

\subsection{Terminology}
\begin{description}
\item[Elliptic-curve cryptography (ECC)]
An approach to public-key cryptography based on the algebraic structure of elliptic curves over finite fields.
\item[Curve25519]
A widely used FIPS-140 approved ECC algorithm not encumbered by any patents \cite{RFC7748}.
\item[Ed25519]
An Edwards-curve Digital Signature Algorithm (EdDSA) over Curve25519 \cite{RFC8032}.
\item[ChaCha20-IETF-Poly1305]
ChaCha20 is a symmetric stream cipher built on a pseudo-random function that gives the advantage that one can
efficiently seek to any position in the key stream in constant time.
It is not patented.
Poly1305 is a cryptographic message authentication code (MAC).
It can be used to verify the data integrity and the authenticity of a message \cite{RFC8439}.

\item[cipher-text]~\\
The encrypted version of the data.

\item[plain-text]~\\
The unencrypted version of the data.

\end{description}

\section{Encrypted Representation Overview}
The encrypted file consists of three parts:

\begin{itemize}
\item An unencrypted header, containing a magic number, version number, and parameters needed to decode the encrypted
header.

\item An encrypted header, which is encrypted using an asymmetric encryption algorithm.
It lists the encryption key and other parameters needed to decrypt the encrypted data section.

\item The encrypted data.
This is the actual application data.
It is encrypted using a symmetric encryption algorithm as described in the encrypted header.
\end{itemize}

\section{Detailed Specification}
\subsection{Overall Conventions}
\subsubsection{Hexadecimal Numbers}
Hexadecimal values are written using the digits 0-9, and letters a-f for values 10-15.
Values are written with the most-significant digit on the left, and prefixed with "0x".

\subsubsection{Byte Ordering}
The basic data size is the byte (8 bits).
All multi-byte vales are stored in least-significant byte first (``little-endian'') order.
For example, the value 1234 decimal (0x4d2) is stored as the byte stream 0xd2 0x04.

\subsubsection{Integer Types}
Integers can be either signed or unsigned.
Signed values are stored in two's complement form.

\subsubsection{Multi-byte Integer Types}
\begin{center}
\begin{tabular}{l l l l}
\hline
\textbf{Name} & \textbf{Byte Ordering} & \textbf{Integer Type} & \textbf{Size (bytes)} \\
\hline
byte & & unsigned & 1 \\
le\_int32 & little-endian & signed & 4 \\
le\_uint32 & little-endian & unsigned & 4 \\
le\_int64 & little-endian & signed & 8 \\
le\_uint64 & little-endian & unsigned & 8 \\
le\_uint96 & little-endian & unsigned & 12 \\
\end{tabular}
\end{center}

\subsubsection{Structures}
Structure types may be defined (in C-like notation) for convenience.

\begin{verbatim}
struct demo {
  byte string[8];
  le_int32 number1;
  le_uint64 number2;
};
\end{verbatim}

When structures are serialized to a file, elements are written in the given order with no padding between them.
The above structure would be written as twenty bytes - eight for the array `string', four for the integer `number1',
and eight for the integer `number2'.

\subsubsection{Enumerated Types}
Enumerated types may only take one of a given set of values.
The data type used to store the enumerated value is given in angle brackets after the type name.
Every element of an enumerated type must be assigned a value.
It is not valid to compare values between two enumerated types.

\begin{verbatim}
enum Animal<le_uint32> {
  cat    = 1;
  dog    = 2;
  rabbit = 3;
};
\end{verbatim}

\subsubsection{Variants}
Parts of structures may vary depending on information available at the time of decoding.
Which variant to use is selected by an enumerated type.
There must be a case for every possible enumerated value.
Cases have limited fall-through.
Consecutive cases with no fields in between all contain the same fields.

\begin{verbatim}
struct AnimalFeatures {
  select (enum Animal) {
    case cat:
    case dog:
      le_uint32 hairyness;
      le_uint32 whisker_length;

    case rabbit:
      le_uint32 ear_length;
  };
};
\end{verbatim}

For the `cat' and `dog' cases, `struct AnimalFeatures' is eight bytes long and contains two unsigned four-byte
little-endian values.
For the `rabbit' case it is four bytes long and contains a single four-byte little-endian value.

If the cases are different lengths (as above), then the size of the overall structure depends on the variant chosen.
There is NO padding to make the cases the same length unless it is explicitly defined.

\subsection{Header}

The file starts with an header, with the following structure:

\begin{verbatim}
struct Header {
  byte                          magic_number[8];
  le_uint32                     version;
  le_uint32                     header_length;
  Header_encryption_parameters  header_encryption;
  byte[]                        encrypted_header
};
\end{verbatim}

The \kw{magic\_number} is the ASCII representation of the string ``crypt4gh''.

The version number is stored as a four-byte little-endian unsigned integer.
The current version number is 1.

The current byte representation of the magic number and version is:
\begin{verbatim}
0x63 0x72 0x79 0x70 0x74 0x34 0x67 0x68 0x01 0x00 0x00 0x00
============= magic_number============= ===== version =====
\end{verbatim}

\kw{header\_length} is the combined length of the \kw{header\_encryption} and \kw{encrypted\_header} sections.

\subsubsection{Header Encryption Parameters}\label{header:encryption}


The `Header\_encryption\_parameters' type is defined as:

\begin{verbatim}

struct Header_encryption_parameters {
  enum Header_encryption_method<le_uint32> header_encryption_method;

  select (header_encryption_method) {
    case X25519_chacha20_ietf_poly1305:
      byte                         public_key[32];
      byte                         nonce[12];
  };
};
\end{verbatim}

\kw{header\_encryption\_method} is an enumerated type that describes the type of encryption to be used.

\kw{public\_key} is the writer's public key.
Treated as a concatenation of eight 32-bit little-endian integers.

\kw{nonce} is a unique initialization vector.
In chacha20\_ietf\_poly1305 it is 12 bytes long.
This value must be unique for each file shared between the same reader and writer.
The best way to ensure this is to generate a value with a cryptographically-secure random number generator.

\subsubsection{Encrypted Header Format}

The encrypted header encodes the parameters used to read the data part of the file.
The plain-text is defined as:

\begin{verbatim}
enum Data_encryption_method<le_uint32> {
  chacha20_ietf_poly1305 = 0;
};

struct EncryptionParameters {
  enum EncryptionMethod<le_uint32>  data_encryption_method;
  select (method) {
    case chacha20_ietf_poly1305:
      byte                          key[32];
  };
};
\end{verbatim}

\kw{data\_encryption\_method} is an enumerated type that describes the type of encryption to be used.

\kw{key} is the symmetric key used to decode the data section.

\subsubsection{X25519\_chacha20\_ietf\_poly1305 Encryption}

This method uses Diffie-Hellman key exchange with additional hashing to generate a shared key.
The shared key is then used along with a randomly-generated nonce to encrypt the data using the chacha20\_ietf\_poly1305
construction.

Encryption requires the writer's public and secret keys ($K_{pw}$ and $K_{sw}$), the reader's public key ($K_{pr}$)
and the nonce ($N$).

The writer's secret key and the reader's public key are used to generate a Diffie\_Hellman shared key using the
X25519 algorithm described in section 5 of \cite{RFC7748}.

$K_{dh} = X25519(K_{sw}, K_{pr})$

As the X25519 algorithm does not produce a completely uniform bit distribution, and many possible $(K_{sw}, K_{pr})$
pairs can produce the same output, the Diffie-Hellman key is hashed along with the two public keys to produce the
final shared key.
The hash function used to do this is Blake2b, as described in \cite{RFC7693}.

$K_s = Blake2b(K_{dh} || K_{pr} || K_{pw})$

As ChaCha20 uses a 32-byte key, only the first 32 bytes of $K_s$ are used; the rest are discarded.

The header data is then encrypted using the method described in the ChaCha20\_IETF\_Poly1305 Encryption
section~\ref{data:chacha20_encryption}.

Finally, the writer's public key $K_{pw}$, the nonce $N$ and the encrypted header data are combined into
the file header.

\subsubsection{X25519\_chacha20\_ietf\_poly1305 Decryption}

To decrypt the header, the reader needs the writer's public key $K_{pw}$ and the nonce from the file header.
Also needed are the reader's public and secret keys ($K_{pr}$ and $K_{sr}$).

The Diffie-Helman key is obtained using:

$K_{dh} = X25519(K_{sr}, K_{pw})$

This is then hashed to obtain the shared key (again only the first 32 bytes are retained):

$K_s = Blake2b(K_{dh} || K_{pr} || K_{pw})$

The resulting key $K_s$ and nonce $N$ are then used to decrypt the remainder of the header.

\subsection{Encrypted Data}
\subsubsection{ChaCha20\_IETF\_Poly1305 Encryption}\label{data:chacha20_encryption}

ChaCha20 is a stream cipher which maps a 256-bit key, nonce and counter to a 512-bit keystream block.
In IETF mode the nonce is 96 bits long and the counter is 32 bits.
The counter starts at 1, and is incremented by 1 for each successive keystream block.
The cipher-text is the plain-text message combined with the keystream using the bitwise exclusive-or operation.

Poly1305 is used to generate a 16-byte message authentication code (MAC) over the cipher-text.
As the MAC is generated over the entire cipher-text it is not possible to authenticate partially decrypted data.

ChaCha20 and Poly1305 are combined using the AEAD construction described in section 2.8 of \cite{RFC8439}.
This construction allows additional authenticated data (AAD) to be included in the Poly1305 MAC calculation.
For the purposes of this format, the AAD is zero bytes long.

\subsubsection{Segmenting the input}

To allow random access without having to authenticate the entire file, the plain-text is divided into 65536-byte
(64KiB) segments.
If the plain-text is not a multiple of 64KiB long, the last segment will be shorter.
Each segment is encrypted using the method defined in the header.
The nonce used to encrypt the segment is then stored, followed by the encrypted data, and then the MAC.

\begin{verbatim}
struct Segment {
  select (method) {
    case chacha20_ietf_poly1305:
      byte    nonce[12];
      byte[]  encrypted_data;
      byte    mac[16];
  };
};
\end{verbatim}

The addition of the \kw{nonce} and \kw{mac} bytes will expand the data slightly.
For chacha20\_ietf\_poly1305, this expansion will be 28 bytes, so a 65536 byte plain-text input will become a 65564
byte encrypted and authenticated cipher-text output.

\section{Decryption}

The cipher-text is decrypted by authenticating and decrypting the segment(s) enclosing the requested byte range $[P;Q]$,
where $P<Q$.
For a range starting at position $P$, the location of the segment \kw{seg\_start} containing that position must
first be found.
For the chacha20\_ietf\_poly1305 method, this can be done using the formula:

\begin{verbatim}
seg_start = header_len + floor(P/65536) * 65564
\end{verbatim}

For an encrypted segment starting at position \kw{seg\_start}, the nonce, then the 65536 bytes of ciphertext
(possibly fewer of it was the last segment), and finally the MAC are read.

An authentication tag is calculated over the ciphertext from that segment, and bitwise compared to the MAC.
The ciphertext is authenticated if and only if the tags match.
An error MUST be reported if the ciphertext is not authenticated.

The key and nonce are then used to decrypt the cipher-text for the segment, returning the plain-text.
Successive segments are decrypted, until the segment containing position $Q$ is reached.
The plaintext segments are concatenated to form the resulting output, discarding $P \mathbin{\%} 65536$ bytes from
the beginning of the first segment and retaining $Q \mathbin{\%} 65536$ bytes of the last one.

\section{Security Considerations}
\subsection{Threat Model}
This format is designed to protect files at rest and in transport from accidental disclosure.
Using authenticated encryption in individual segments mirrors solutions like Transport Layer
Security (TLS) as described in~\cite{RFC5246} and prevents undetected modification of segments.

\subsection{Selection of Key}
The security of the format depends on attackers not being able to guess the encryption key.
The encryption key MUST be generated using a cryptographically-secure pseudo-random number generator.
This makes the chance of guessing a key vanishingly small.

\subsection{Nonce selection}
All segments encrypted with the same key MUST use a unique nonce.
One way to ensure this is to choose a random starting point and then use a counter or linear-feedback shift
register to generate the nonce for each segment.
This method guarantees a unique value even for very long files.

For a 96-bit nonce, it may be acceptable to generate each one using a cryptographically-secure pseudo-random number
generator.
Care should be taken to ensure that the random number generator is capable of generating a long enough stream
of unique values.
Due to the birthday problem, this method will have a non-zero (but very small) probability of failing.
For example, a file of 24 Terabytes will have a reused nonce with probability of approximately $10^-12$

\subsection{Message Forgery}
Using Curve25519 and Ed25519 authenticates the content of the encrypted file header.
Using ChaCha20-ietf-Poly1305 authenticates the content of each segment of the encrypted cipher-text.

\subsection{No File Updates Permitted}
Implementations MUST NOT update encrypted files.
Once written, a section of the file must never be altered.

\addcontentsline{toc}{section}{References}
\bibliographystyle{abstract}

\renewcommand{\refname}{\section{References}} % Get References numbered correctly in the table of contents
\begin{thebibliography}{XXXXXXX}

% Normative references

\bibitem[RFC2119]{RFC2119}
  Bradner, S.,\\
  \emph{"Key words for use in RFCs to Indicate Requirement Levels", BCP 14, RFC 2119},\\
  \url{https://www.rfc-editor.org/info/rfc2119},\\
  March 1997

\bibitem[RFC7693]{RFC7693}
  Saarinen, M.J., Aumasson, J.P.\\
  \emph{"The BLAKE2 Cryptographic Hash and Message Authentication Code (MAC)", RFC7693},\\
  \url{https://tools.ietf.org/html/rfc7693},\\
  November 2015

\bibitem[RFC7748]{RFC7748}
  A. Langley, Google, M. Hamburg, Rambus Cryptography Research, S. Turner, sn3rd,\\
  \emph{"Elliptic Curves for Security", RFC7748},\\
  \url{https://tools.ietf.org/html/rfc7748},\\
  January 2016

\bibitem[RFC8032]{RFC8032}
  S. Josefsson, SJD AB, I. Liusvaara,\\
  \emph{"Edwards-Curve Digital Signature Algorithm (EdDSA)", RFC8032},\\
  \url{https://tools.ietf.org/html/rfc8032},\\
  January 2017

\bibitem[RFC8439]{RFC8439}
  Y. Nir, Dell EMC, A. Langley, Google, Inc.,\\
  \emph{"ChaCha20 and Poly1305 for IETF Protocols", RFC8439},\\
  \url{https://tools.ietf.org/html/rfc8439},\\
  June 2018

% Informational references

\bibitem[RFC5246]{RFC5246}
  Dierks, T., Rescorla, E.,\\
  \emph{"The Transport Layer Security (TLS) Protocol Version 1.2", RFC 5246},\\
  \url{https://www.rfc-editor.org/info/rfc5246},\\
  August 2008.

\end{thebibliography}

\end{document}

\title{GA4GH File Encryption Standard}
\author{Robert Davies}
\date{\headdate
\maketitle
\begin{quote}\small
The master version of this document can be found at
\url{https://github.com/samtools/hts-specs}.\\
This printing is version~\commitdesc\ from that repository,
last modified on the date shown above.
\end{quote}
\vspace*{1em}

\begin{abstract}
This document describes the format for Global Alliance for Genomics and Health (GA4GH) encrypted files.
Encryption helps to prevent accidental disclosure of confidential information.
Allowing programs to directly read and write data in an encrypted format reduces the chance of such disclosure.
The format described here can be used to encrypt any underlying file format.
It also allows for seeking on the encryted data.
In particular indexes on the plain text version can also be used on the encrypted file without modification.
\end{abstract}
\newpage
\section*{Copyright Notice}
Copyright \copyright 2017 Genome Research Limited.  All rights reserved.

\tableofcontents
\newpage
\section{Introduction}
\subsection{Purpose}
By its nature, genomic data can include information of a confidential nature about the health of individuals.
It is important that such information is not accidentally disclosed.
One part of the defense against such disclosure is to, as much as possible, keep the data in an encrypted format.

This document describes a file format that can be used to store data in an encrypted state.
Existing applications can, with minimal modification, read and write data in the encrypted format.
The choice of encryption also allows the encrypted data to be read starting from any location, facilitating indexed
access to files.

\subsection{Requirements}
The key words ``MUST'', ``MUST NOT'', ``REQUIRED'', ``SHALL'', ``SHALL NOT'', ``SHOULD'', ``SHOULD NOT'',
``RECOMMENDED'', ``MAY'', and ``OPTIONAL'' in this document are to be interpreted as described in \cite{RFC2119}.

\subsection{Terminology}
\begin{description}
\item[Advanced Encryption Standard (AES)]~\\
A FIPS-approved algorithm that can be used to protect data, defined in
\cite{AESSTD}.
AES encrypts data in fixed-sized blocks of 16 bytes.

\item[cipher-text]~\\
The encrypted version of the data.

\item[counter (CTR) mode]~\\
A method of using a cipher with a fixed block size to encrypt data streams that are longer than a single block,
converting it to a pseudo-random block of bytes.
The counter is incremented for each block, leading to the generation of a repeatable byte stream.
This stream can be XORed with the plain-text to generate the cipher-text.
Because XOR is reversible, the same operation can also be used to convert the cipher-text back into the plain-text.

CTR mode and its use with AES are described in \cite{CTRMODE}.

\item[plain-text]~\\
The unencrypted version of the data.
 
\end{description}

\section{Encrypted Representation Overview}
The encrypted file consists of three parts:

\begin{itemize}
\item An unencrypted header, containing a magic number, version number, and the length of the encrypted header

\item An encrypted header.  This is encrypted using an asymmetric encryption algorithm.
It lists the encryption keys and initialization vectors needed to decrypt the encrypted data section.

\item The encrypted data.
This is the actual application data.
It is encrypted using a symmetric encryption algorithm as described in the encrypted header.
\end{itemize}

\section{Detailed Specification}
\subsection{Overall Conventions}
\subsubsection{Hexadecimal Numbers}
Hexadecimal values are written using the digits 0-9, and letters a-f for values 10-15.
Values are written with the most-significant digit on the left, and prefixed with "0x".

\subsubsection{Byte Ordering}
The basic data size is the byte (8 bits).
Multi-byte values need to be stored in a defined order.
This order will be one of the following:
\begin{description}
\item[Least-significant byte first (``little-endian'')]~\\
The value 1234 decimal (0x4d2) is stored as the byte stream 0xd2 0x04.

\item[Most-significant byte first (``big-endian'')]~\\
The value 1234 decimal (0x4d2) is stored as the byte stream 0x04 0xd2.
\end{description}

\subsubsection{Integer Types}
Integers can be either signed or unsigned.
Signed values are stored in two's complement form.

\subsubsection{Multi-byte Integer Types}
\begin{center}
\begin{tabular}{l l l l}
\hline
\textbf{Name} & \textbf{Byte Ordering} & \textbf{Integer Type} & \textbf{Size (bytes)} \\
\hline
byte & & unsigned & 1 \\
le\_int32 & little-endian & signed & 4 \\
le\_uint32 & little-endian & unsigned & 4 \\
le\_int64 & little-endian & signed & 8 \\
le\_uint64 & little-endian & unsigned & 8 \\
be\_uint128 & big-endian & unsigned & 16 \\
\end{tabular}
\end{center}

\subsubsection{Vectors}
A vector is a stream of elements of the same type (which may be a structure).
The number of items may be specified either as a constant value or in reference to a known integer value (for
example, a variable previously read from the file).

\begin{verbatim}
le_in32   num     // Number of v2 array elements
le_int32  v1[8]   // Eight four-byte little-endian integers
le_int64  v2[num] // 'num' eight-byte little-endian integers
\end{verbatim}

When vectors are serialized to a file, the elements are written with no padding between them.

\subsubsection{Structures}
Structure types may be defined for convenience.
The syntax for definition is similar to that of C.

\begin{verbatim}
struct demo {
  byte string[8];
  le_int32 number1;
  le_uint64 number2;
};
\end{verbatim}

When structures are serialized to a file, elements are written in the given order with no padding between them.
So the structure above would be written as twenty bytes - eight for the array `string', four for the integer `number1',
and eight for the integer `number2'.

\subsubsection{Enumerated Types}
Enumerated types may only take one of a given set of values.
The data type used to store the enumerated value is given in angle brackets after the type name.
Every element of an enumerated type must be assigned a value.
It is not valid to compare values between two enumerated types.

\begin{verbatim}
enum Animal<le_uint32> {
  cat    = 1;
  dog    = 2;
  rabbit = 3;
};
\end{verbatim}

\subsubsection{Variants}
Parts of structures may vary depending on information available at the time of decoding.
Which variant to use is selected by an enumerated type.
There must be a case for every possible enumerated value.
Cases have limited fall-through.
Consecutive cases with no fields in between all contain the same fields.

\begin{verbatim}
struct AnimalFeatures {
  select (enum Animal) {
    case cat:
    case dog:
      le_uint32 hairyness;
      le_uint32 whisker_length;

    case rabbit:
      le_uint32 ear_length;
  };
};
\end{verbatim}

For the `cat' and `dog' cases, `struct AnimalFeatures' is eight bytes long and contains two unsigned four-byte
little-endian values.
For the `rabbit' case it is four bytes long and contains a single four-byte little-endian value.

If the cases are different lengths (as above), then the size of the overall structure depends on the variant chosen.
There is NO padding to make the cases the same length unless it is explicitly defined.

\subsection{Unencrypted Header}

The file starts with an unencrypted header, with the following structure:

\begin{verbatim}
struct Unencrypted_header {
  byte      magic_number[8];
  le_uint32 version;
  le_uint32 header_len;
};
\end{verbatim}

The magic\_number is the ASCII representation of the string ``crypt4gh''.

The version number is stored as a four-byte little-endian unsigned integer.
The current version number is 1.

hdr\_len is the sum of the lengths of the unencrypted and encrypted headers.
It is stored as a four-byte little-endian unsigned integer.
As it includes the unencrypted header, hdr\_len will always have a value of at least 16.

The current byte representation of the magic number and version is:
\begin{verbatim}
0x63 0x72 0x79 0x70 0x74 0x34 0x67 0x68 0x01 0x00 0x00 0x00
============= magic_number============= ===== version =====
\end{verbatim}

\subsection{Encrypted Header}
\subsubsection{Encryption Method}
The encrypted header is encoded in the OpenPGP message format \cite{RFC4880}.

\subsubsection{Plain-text Format}
The plain-text data encoded in the encrypted header has the following overall structure:
\begin{verbatim}
struct Encrypted_header {
  le_uint32                     number_records;
  struct Encryption_parameters  records[number_records];
};
\end{verbatim}

`number\_records' gives the number of `Encryption\_parameters' records that follow.
Each record corresponds to a range of bytes in the plain-text version of the file.
This allows chunks of encrypted data to be rapidly concatenated together without the need to decrypt and re-encrypt the
whole file.

The `Encryption\_parameters' type is defined as:

\begin{verbatim}
enum Encryption_method<le_uint32> {
  AES-256-CTR = 0;
};

struct Encryption_parameters {
  le_uint64                         plaintext_start;
  le_uint64                         plaintext_end;
  le_uint64                         ciphertext_start;
  le_int64                          counter_offset;

  enum Encryption_method<le_uint32> method;

  select (method) {
    case AES-256-CTR:
      byte       key[32];
      be_uint128 iv;
  };
};
\end{verbatim}

`plaintext\_start' and `plaintext\_end' define the range of plain-text bytes over which this record is valid.
The positions are indexed counting from zero.
plaintext\_end is the byte one past the end of the range.
The value of plaintext\_end is allowed to be beyond the real end of the plain-text data.
This means it can be set to \verb+0xffffffffffffffff+ when the final length of the file is unknown.

`ciphertext\_start' is the location in the encrypted data where the given plain-text data starts.
This location counts from 0 at the start of the encrypted data section.
header\_len from the unencrypted header should be added to this to find the location in the encrypted file.

`counter\_offset' is used to adjust the counter value used in counter mode (CTR) encryption.
This allows the correct counter value to be calculated for files that have been concatenated together.

The regions [plaintext\_start..plaintext\_end) MUST NOT overlap between records.
Likewise, the corresponding cipher-text blocks MUST NOT overlap.
To facilitate streaming, they SHOULD be in order, and the order of blocks in the cipher-text SHOULD match the order
in the plain-text.

The union of regions over all the records does not have to cover the entire plain-text.
This allows selective access to parts of a file to be granted.
When preparing files, inaccessible parts SHOULD NOT be included in the cipher-text.
It wastes the recipient's storage, and readers still may be able to gain access if they are encrypted with the same key
as other parts of the file that are permitted.

`method' is an enumerated type that describes the type of encryption to be used.

`key' is a secret encryption key.

`iv' is an initialization vector.

\subsection{Encrypted Data}
\subsubsection{AES-256-CTR Mode Encryption}

AES-CTR mode generates a byte stream by AES encrypting a sequential counter.
The counter starts at the given IV and increments by 1 for each block of 16 bytes.
This byte stream is XORed with the plain-text to produce the cipher-text.

The cipher-text is decrypted by XORing it with the same byte stream, which can be generated if the IV and encryption
key are known.

In this format, the counter value C for a particular plain-text file offset P can be calculated as follows:
\begin{verbatim}
  C = (P + counter_offset) / 16
\end{verbatim}
where counter\_offset is given the encrypted header record.
The sum (P + counter\_offset) should always be positive.

The counter value is then combined with the IV by addition.
For compatibility with OpenSSL, both C and IV are treated as 16-byte big-endian values.

The sum C + IV and the key are used to generate 16 bytes of key stream E:
\begin{verbatim}
  E[0..15] = AES(key, C + IV)
\end{verbatim}
  This can be used to encrypt the byte B at file offset P:
\begin{verbatim}
  B' = B XOR E[(P + counter_offset) % 16]  where % is the modulo operator.
\end{verbatim}

\subsubsection{Partial Blocks}
It is likely that the end of the file, or the boundaries of a region listed in the encrypted header, will not occur at
an exact multiple of the cipher block length.
In such cases, the unused bytes of the key stream are simply discarded.
There is no need to pad the encrypted data.

\section{Security Considerations}
\subsection{Threat Model}
This format is only designed to protect files at rest from accidental disclosure.
In particular, it is not designed to protect files during transmission over insecure networks - existing solutions like
Transport Layer Security (TLS) as described in \cite{RFC5246} should be used for this.

\subsection{Selection of Keys and IVs}
The security of the format depends on attackers not being able to guess the encryption key (and to a lesser extent the
IV).
The encryption key MUST be generated using a cryptographically-secure pseudo-random number generator.
This makes the chance of guessing a key vanishingly small.

When using CTR mode encryption, it is vital never to reuse an (IV + CTR) value for the same cryptographic key.
Doing so could lead to two blocks being encrypted with identical key streams.
If this happens, then an attacker can learn the XOR of the plain-text of the two blocks by simply XORing the cipher-text
blocks (as key\_stream XOR key\_stream = 0).
Values of (IV + CTR) MUST NOT be reused for the same key.
This can be achieved by:

\begin{itemize}
\item Choosing a random IV using a cryptographically-secure random number generator.
\item Always incrementing the counter for each block (as happens in this format).
\end{itemize}

If both the cryptographic key and IV are chosen randomly for each file, the chance of reusing a combination of
(key, IV + CTR) becomes vanishingly small.
Implementations SHOULD choose the IV using a cryptographically secure random number generator.

\subsection{Message Forgery}
Forging a message encrypted in CTR mode is trivial, especially if the attacker can guess the plain-text of an encrypted
message.
They simply XOR the cipher-text with a chosen byte stream.
On decoding, the result will be the original plain-text XORed with the attacker's byte stream.

If an attacker knows that part of an encrypted file is, for example, a gzip header, then they know the plain-text bytes
and can replace that part with any data they choose.

This format is currently not resistant to forgery.
Care should be taken to protect files from unauthorised tampering.
This is also why extra precautions should be taken when transmitting files over networks.

\subsection{No File Updates Permitted}
While it is possible to update parts of a file written in CTR mode, it violates the rule about not reusing the value
of (IV + CTR) for a given key.
An attacker who compares the before and after versions of the file will learn the XOR of the before and after
plain-texts.

Implementations MUST NOT update encrypted files.
Once written, a section of the file must never be altered.

\section{References}

\begin{thebibliography}{XXXXXXX}

% Normative references

\bibitem[AESSTD]{AESSTD}
  National Institute of Standards and Technology,\\
  \emph{"Announcing the ADVANCED ENCRYPTION STANDARD (AES)", FIPS 197},\\
  \url{http://csrc.nist.gov/publications/fips/fips197/fips-197.pdf},\\
  November 26, 2001.

\bibitem[RFC2119]{RFC2119}
  Bradner, S.,\\
  \emph{"Key words for use in RFCs to Indicate Requirement Levels", BCP 14, RFC 2119},\\
  \url{https://www.rfc-editor.org/info/rfc2119},\\
  March 1997.

\bibitem[RFC4880]{RFC4880}
  Callas, J., Donnerhacke, L., Finney, H., Shaw D., Thayer, R.,\\
  \emph{``OpenPGP Message Format'', RFC4880},\\
  \url{https://www.rfc-editor.org/info/rfc4880},\\
  November 2007.

% Informational references

\bibitem[CTRMODE]{CTRMODE}
  Dworkin, M, National Institute of Standards and Technology,\\
  \emph{"Recommendation for Block Cipher Modes of Operation", SP800-38A},\\
  \url{http://nvlpubs.nist.gov/nistpubs/Legacy/SP/nistspecialpublication800-38a.pdf},\\
  December 2001.

\bibitem[RFC5246]{RFC5246}
  Dierks, T., Rescorla, E.,\\
  \emph{"The Transport Layer Security (TLS) Protocol Version 1.2", RFC 5246},\\
  \url{https://www.rfc-editor.org/info/rfc5246},\\
  August 2008.

\end{thebibliography}

\end{document}
